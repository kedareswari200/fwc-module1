\def\mytitle{ Design of XNOR Gate Using NOR Gates}
\def\mykeywords{}
\def\myauthor{SOMISETTY KEDARESWARI}
\def\contact{mail2kedari@gmail.com}
\def\mymodule{ Future Wireless Communication(FWC22049)}
% #######################################
% #### YOU DON'T NEED TO TOUCH BELOW ####
% #######################################
\documentclass[10pt, a4paper]{article}
\usepackage[a4paper,outer=1.5cm,inner=1.5cm,top=1.75cm,bottom=1.5cm]{geometry}
\twocolumn
\usepackage{circuitikz}
\usepackage{graphicx}
\graphicspath{{./images/}}
%colour our links, remove weird boxes
\usepackage[colorlinks,linkcolor={black},citecolor={blue!80!black},urlcolor={blue!80!black}]{hyperref}
%Stop indentation on new paragraphs
\usepackage[parfill]{parskip}
%% Arial-like font
\usepackage{lmodern}
\renewcommand*\familydefault{\sfdefault}
%Napier logo top right
\usepackage{watermark}
%Lorem Ipusm dolor please don't leave any in you final report ;)
\usepackage{karnaugh-map} 
\usepackage{tabularx}
\usepackage{lipsum}
\usepackage{xcolor}
\usepackage{listings}
%give us the Capital H that we all know and love
\usepackage{float}
%tone down the line spacing after section titles
\usepackage{titlesec}
%Cool maths printing
\usepackage{amsmath}
%PseudoCode
\usepackage{algorithm2e}

\titlespacing{\subsection}{0pt}{\parskip}{-3pt}
\titlespacing{\subsubsection}{0pt}{\parskip}{-\parskip}
\titlespacing{\paragraph}{0pt}{\parskip}{\parskip}
\newcommand{\figuremacro}[5]{
    \begin{figure}[#1]
        \centering
        \includegraphics[width=#5\columnwidth]{#2}
        \caption[#3]{\textbf{#3}#4}
        \label{fig:#2}
    \end{figure}
}


 \lstset{
frame=single, 
breaklines=true,
columns=fullflexible
}

\thiswatermark{\centering \put(1,-110){\includegraphics[scale=0.05]{IIT Hyd.png}} }
\title{\mytitle}
\author{\myauthor\hspace{1em}\\\contact\\IITH\hspace{0.5em}-\hspace{0.5em}\mymodule}
\date{}
\hypersetup{pdfauthor=\myauthor,pdftitle=\mytitle,pdfkeywords=\mykeywords}
\sloppy
% #######################################
% ########### START FROM HERE ###########
% #######################################
\begin{document}
 \maketitle

 \begin{abstract}
     %Replace the lipsum command with actual text 
  We can able to design all other gates using the pair Universal gates i.e;(NAND and NOR).This document is to understand the behavior and demonstrate the Implementation of XNOR Gate using NOR gate.
 \end{abstract}
    
\section{Components}
\begin{table}[htbp]
 \begin{center}
    \begin{tabular}{|l|c|c|c|c|} \hline
  \textbf{Component}& \textbf{Value} & \textbf{Quantity} \\
 \hline
 bread board& - & 1 \\ \hline
led &  - & 1 \\ \hline
Arduino & - & 1 \\ \hline
Jumper Wires & M-M & 2  \\ \hline
\end{tabular}  
\end{center}
\caption{\label{table:dummytable} }
\end{table}



\section{XNOR Truth Table}
\begin{table}[htbp]
 \begin{center}
    \begin{tabular}{|l|c|c|c|c|} \hline
  \textbf{A}& \textbf{B} & \textbf{G(A,B)} \\
 \hline
 0&0&1\\ \hline
0&1&0 \\ \hline
1&0&0\\ \hline
1&1&1\\ \hline
\end{tabular}  
\end{center}
\caption{\label{table:dummytable} }
\end{table}

\section{Circuit Diagram}
\begin{circuitikz} \draw
(-1,1) node[nor port] (nor1) {}
(nor1.in 1) -- ++ (-1,0) node[circ]{} node[left]{$A$}
(nor1.in 2) -- ++ (-1,0) node[circ]{} node[left]{$B$}
(-1,1) node[nor port] (nor1) {}
(nor1.out) node(x1) [anchor=south west]{$x1$}
(1,2) node[nor port] (nor2) {}
(1,0) node[nor port] (nor3) {}
(3,1) node[nor port] (nor4) {}


(1,2) node[nor port] (nor2) {}
(nor2.out) node(x2) [anchor=south west]{$x2$}
(1,0) node[nor port] (nor3) {}
(nor3.out) node(x3) [anchor=north west] {$x3$}
(3,1) node[nor port] (nor4) {}
(nor4.out) node(x4) [anchor=south west] {$x4$}




(nor2.out) -- (nor4.in 1)
(nor3.out) -- (nor4.in 2)
(nor1.out) -| (nor2.in 2)
(nor1.out) -| (nor3.in 1)
(nor1.in 1) |-(nor2.in 1)
(nor1.in 2) |-(nor3.in 2)
\end{circuitikz}

\\ \begin{center}
\begin{center}
    Figure 1
\end{center}

\end{center}

 
     
    \section{Boolean Logic}
     x1=(A+B)'\\
   x2=(A+x1)'\\ 
   x3=(B+x1)'\\
   x4=(x2+x3)' \\
   
    \section{Hardware}
    
\begin{table}[htbp]
\begin{center}
\begin{tabular}{|l|c|c|c|c|c|c|} \hline 
\textbf{Arduino} & \textbf{D13} & \textbf{GND} \\ \hline
\textbf{Led} & \textbf{+VE} & \textbf{-VE}\\ \hline
\end{tabular}   
\end{center}
\caption{\label{table:dummytable}}
\end{table}
   
   \section{Hardware Connection}
   
Give the connections as per Table 3. For taking the inputs connect 5V of arduino to +ve line of bread board to consider it as logic 'HIGH'.Connect GND pin of arduino to -ve line of bread board to consider it as logic 'LOW'.
\\
\\
For example if the inputs A,B are connected 1,0 respectively the output should be 0 i.e., the LED connected to the 13th pin should turn off.
\\
\\
In the another case if we connect the inputs A,B to 1,1 respectively the output should be 1 i.e., the LED connected to 13th pin should glow.

The circuit implementation of the above function is given in figure 1.








  \section{Software}
  1.Connect the arduino to the USB port of computer
  \\
  \\2.Download the follwing code
  \\
  \begin{lstlisting}
     https://github.com/kedareswari200/fwc-module1/blob/main/codes/src/main.cpp
  \end{lstlisting}
  
  3.Upload the code into the arduino board.
  \\
  \\4.The output '1' is represented as the state:'LED ON' and '0' is represented as the state 'LED OFF' 
 
  


\end{document}

    
